\acknowledgement
本次课程设计基于接触网工程设计的基本原理,结合初始设计资料和相关技术标准,对选定车站线路的接触网进行了初步工程设计。在设计的过程中,首先进行了接触网的机械计算,包括负荷计算、最大跨距确定、半补偿链型悬挂的安装曲线以及锚段长度和张力增量曲线的计算等关键步骤。

接下来,基于相关技术资料,详细讨论了接触网的平面设计,包括支柱布置、拉出值的大小和方向、锚段关节设计以及锚段的分段等内容。这些设计考虑了站场的特殊情况,并利用CAD软件完成了站场平面设计图和咽喉区放大图的绘制,确保设计方案的可行性和准确性。

最后,对软横跨支柱容量和缓和曲线接触线的最大偏移进行了校验,并得出了符合设计规范的结论。通过这次课程设计,我将接触网工程理论与实际设计施工相结合,深化了对接触网设计的理解,为未来的学习和工作奠定了坚实的基础。此外,我还自学了MATLAB编程计算和作图,以及CAD平面设计,提升了实际操作能力,这是一次非常有意义的学习经历。